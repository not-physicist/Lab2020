\clearpage
\section{Preparatory Tasks}
\paragraph{Calculation of the deflection potential $\psi(\theta) $ and the scaled deflection angle of an SIS lens.}\hspace{0pt}\\
 From equation \ref{psiTheta} the deflection potential, which gives the information about the mass distribution of the lens, is define as,
 \begin{equation}
 \psi(\theta)=\frac{1}{\pi}\int d^2\theta'\kappa(\theta')\ln |\theta-\theta'| 
 \end{equation}
 In the case of axial symmetry of SIS lens equation \ref{psiTheta} simplifies to (given in the question)
 \begin{equation}
 \psi(\theta)=2\int_{0}^{\theta}d\theta' \theta'\kappa(\theta') \ln \bigg(\frac{\theta}{\theta'}\bigg)
 \label{Equ:psiSIS}
 \end{equation}

 By substituting the values of $ \Sigma(D_{d}\theta) $, where $ D_{d}\theta=\xi $ from equation \ref{equ:Sigma(xi)} and  $ \Sigma_\text{cr} $ from equation \ref{SumCr} and equation \ref{Equ:KTheta}, then by plugging the new expression of  $ \kappa(\theta) $ equation \ref{Equ:psiSIS} becomes
 \begin{equation*}
	 \psi(\theta) =\frac{4\pi}{c^2}\frac{D_{ds}}{D_{s}}\sigma^2_{v}\int_{0}^{\theta}d\theta' \ln \bigg(\frac{\theta}{\theta'}\bigg) \\
 \end{equation*}
 Therefore for $\theta > 0$, 
 \begin{equation}
 \psi(\theta)=\theta_{E}\theta 
 \label{Equ:psiFinal}
 \end{equation}
 From equation \ref{Equ:AlphaTheta} scaled deflection angle is defined as
 \begin{equation*}
 \alpha(\theta)=\nabla\psi(\theta)
 \label{Equ:AlphaTheta2}
 \end{equation*}
 From equations: \ref{Equ:psiFinal} and \ref{Equ:AlphaTheta2},
 \begin{equation}
 \alpha(\theta)=\nabla \theta_{E}\theta = \theta_{E} \frac{\theta}{|\theta|}
 \label{Equ:AlphaThetaFinal}
 \end{equation}
 
 \paragraph{Solving the lens equation and finding the separation between images}\hspace{0pt} \\
 The lens equation in axial symmetric case can be written as
 \begin{equation*}
 \beta=\theta-\alpha(\theta)
 \end{equation*}
 By using the expression for scaled deflection angle for SIS from \ref{Equ:AlphaThetaFinal}
 \begin{equation*}
 \beta=\theta-\theta_{E} \frac{\theta}{|\theta|}
 \end{equation*}
 % or,
 % \begin{equation*}
	 % \alpha(\theta)=\beta+\theta_{E}\frac{\theta}{|\theta|}
 % \end{equation*}

 There are two solutions ($\theta/|\theta| = \pm 1$)
 \begin{equation}
 \theta_{A, B}=\beta \pm \theta_{E}
 \label{Equ:ThetaAB}
 \end{equation}
 Thus, the separation of these  images is given by
 \begin{equation}
 \Delta\theta= \theta_{A}- \theta_{B} = 2\theta_{E}
 \label{math:imSep}
 \end{equation}
 
 \paragraph{Magnification ratio of the two images of SIS lens.}\hspace{0pt}\\
 The determinant can be computed with equation~\ref{Equ:ThetaAB} by
 \begin{equation}
	 \det \mathcal{A} = \frac{\beta}{\theta}\dv{\beta}{\theta} 
							= 
							\begin{cases}
								1 - \frac{\theta_E}{\theta_A} & \text{for A} \\
								1 + \frac{\theta_E}{\theta_B} & \text{for B}
							\end{cases}
 \end{equation}

 The magnification of a gravitational lens is given by equation~\ref{math:mag}. If one assume the ratio of magnification is the ratio of flux, then image A has more flux coming in. The flux ratio of two images is
 \begin{equation}
	 \left| \frac{\mu_A}{\mu_B} \right| = \frac{\theta_A}{\theta_B}
 \end{equation}
 
 \paragraph{Time delay derivation for SIS lens as a function of $ \theta_{A}$ and $ \theta_{B}$.}\hspace{0pt}\\
 Plugin the solution in equation~\ref{Equ:TimeDelay} and also taking care of negative image, we have
 \begin{equation}
	 c \Delta t_\text{SIS} = \frac{1}{2} (1 + z_d) \frac{D_d D_s}{D_{ds}} (\theta_A^2 -\theta_B^2)
	 \label{math:timeDelaySIS}
 \end{equation}
  
  
 \paragraph{Minimum Dispersion Estimator}\hspace{0pt} \\
  Minimum dispersion method as described before assume the functional form of light curves is sufficiently smooth, so that the function in between the increments of $\lambda$ and $\Delta m$ can be properly interpolated. One way to deal with this, is to explore the whole 2d plane of $D^2(\lambda, \Delta m)$.
  
  
  \paragraph{The approximation of dispersion function near the minimum by a parabola.}\label{sec:parabola}\hspace{0pt}\\
  The dispersion function near the minimum can be found by Taylor expanding equation~\ref{math:disp}
  \begin{equation}
	  D^2(\lambda)=D^2(\lambda_{0}) +\frac{\text{d}D^2(\lambda_{0})}{\text{d}\lambda}(\lambda- \lambda_{0}) + \frac{\text{d}^2 D^2(\lambda_{0})}{\text{d}\lambda^2}(\lambda - \lambda_{0})^2 +  \order{(\lambda-\lambda_0)^3}
  \label{Equ:lambda}
  \end{equation}
  The dispersion function is minimum at $ \lambda= \lambda_{0} $. So the second term of equation \ref{Equ:lambda} goes to zero.
 
