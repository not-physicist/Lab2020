\section{Preparatory Tasks}
\textbf{ P.3.1 Calculation of the deflection potential, $\psi(\theta) $ and the scaled deflection angle of an SIS lens.}\\
 
 \noindent
 From equation \ref{psiTheta} the deflection potential, which gives the information about the mass distribution of the lens, is define as,
 \begin{equation}
 \psi(\theta)=\frac{1}{\pi}\int d^2\theta'\kappa(\theta')In \mid \theta-\theta'\mid 
 \end{equation}
 In the case of axial symmetry of SIS lens equation \ref{psiTheta} simplifies to (Given on the question)
 \begin{equation}
 \psi(\theta)=2\int_{0}^{\theta}d\theta' \theta'\kappa(\theta')ln \bigg(\frac{\theta}{\theta'}\bigg)
 \label{Equ:psiSIS}
 \end{equation}

 By substituting the values of $ \Sigma(D_{d}\theta) $, where $ D_{d}\theta=\xi $ from equation \ref{equ:Sigma(xi)} and  $ \Sigma_{cr} $ from equation \ref{SumCr} on equation \ref{Equ:KTheta}, then by plugging the new expression of  $ \kappa(\theta) $ equation \ref{Equ:psiSIS} becomes
 
 
 \begin{equation}
 \psi(\theta)=\frac{4\pi}{c^2}\frac{D_{ds}}{D_{s}}\sigma^2_{v}\int_{0}^{\theta}d\theta' ln \bigg(\frac{\theta}{\theta'}\bigg)
 \end{equation}
 
 By integrating,
 \begin{equation}
 =\frac{4\pi}{c^2}\frac{D_{ds}}{D_{s}}\sigma^2_{v}[\theta'ln(\frac{\theta}{\theta'})+\theta']
 \end{equation}
 
 \begin{equation}
 =\theta_{E}(\theta ln \theta - \theta ln \theta + \theta )
 \end{equation}
 
 Therefore, 
 \begin{equation}
 \psi(\theta)=\theta_{E}\theta 
 \label{Equ:psiFinal}
 \end{equation}
 From equation \ref{Equ:AlphaTheta} scaled deflection angle is defined as
 \begin{equation}
 \alpha(\theta)=\bigtriangledown\psi(\theta)
 \label{Equ:AlphaTheta2}
 \end{equation}
 From equations: \ref{Equ:psiFinal} and \ref{Equ:AlphaTheta2},
 \begin{equation}
 \alpha(\theta)=\bigtriangledown \theta_{E}\theta = \theta_{E}\hat{\theta}
 \label{Equ:AlphaThetaFinal}
 \end{equation}
 \\
 
 \textbf{P.3.2: Solving the lens equation and finding the separation between images}\\
 
 The lens equation is given by,
 \begin{equation}
 \beta=\theta-\alpha(\theta)
 \end{equation}
 By using the expression for scaled deflection angle for SIS from \ref{Equ:AlphaThetaFinal}
 \begin{equation}
 \beta=\theta-\theta_{E}\hat{\theta}
 \end{equation}
 or, 
 \begin{equation}
 \alpha(\theta)=\beta+\theta_{E}\hat{\theta}
 \end{equation}
 for $ \hat{\theta}>0 $,
 \begin{equation}
 \theta_{A}=\beta + \theta_{E}
 \label{Equ:ThetaA}
 \end{equation}
 for $ \hat{\theta}<0 $,
 \begin{equation}
 \theta_{B}=\beta -\theta_{E}
 \label{Equ:ThetaB}
 \end{equation}
 Thus, the separation of these  images is given by,
 \begin{equation}
 \Delta\theta= \theta_{A}- \theta_{B} = \beta+\theta_{E}-(\beta-\theta_{E})= 2\theta_{E}
 \end{equation}
 \\
 
 \textbf{P.3.3: Magnification ratio of the two images of SIS lens.}\\
 
 The magnification of a gravitational lens is given by,
 \begin{equation}
 \mu=(det \text{A})^{-1}
 \end{equation}\\
 
 
 
 
 
 \textbf{P.3.4: Time delay derivation for SIS lens as a function of $ \theta_{A}$ and $ \theta_{B}$.}\\
 
 The time delay is given by\cite{manual}
 \begin{equation}
 \text{c}\Delta\text{t}(\beta)=(1+ \text{z}_{d})\frac{\text D_{d}D_{s}}{D_{ds}}[\tau(\theta_{A};\beta) - \tau(\theta_{B}; \beta)]
 \label{Equ:timeDelay}
 \end{equation}
 
 Also, from \ref{Equ:Format} the Format's potential is defined as,
 \begin{equation}
 \tau(\theta; \beta)=\frac{1}{2}(\beta -\alpha)^2 -\psi(\theta)
 \end{equation}
 So,
 
 \begin{equation}
 \tau(\theta_{A}; \beta)=\frac{1}{2}(\beta -\theta_{A})^2 -\psi(\theta_{A})
 \end{equation}
 By plugging the values of $\beta$ and $\psi(\theta_{A})$,
 \begin{equation}
 =\frac{1}{2}(\theta_{A}-\theta_{E} -\theta_{A})^2 -\theta_{E}\theta_{A}
 \end{equation}
 Therefore,
 \begin{equation}
 \tau(\theta_{A}; \beta)=\frac{1}{2}\theta^2_{E} -\theta_{E}\theta_{A}
 \end{equation}
 
  Similarly,
  \begin{equation}
  \tau(\theta_{B}; \beta)=\frac{1}{2}(\beta -\theta_{B})^2 -\psi(\theta_{B})
  \end{equation}
  
   \begin{equation}
   =\frac{1}{2}(\theta_{B}+\theta_{E} -\theta_{B})^2 -\theta_{E}\theta_{B}
   \end{equation}
 
   \begin{equation}
  \tau(\theta_{B}; \beta) =\frac{1}{2}\theta^2_{E} - \theta_{E}\theta_{B}
   \end{equation}
  By substituting these values equation \ref{Equ:timeDelay} becomes, 
 \begin{equation}
 \text{c}\Delta\text{t}(\beta)=(1+ \text{z}_{d})\frac{\text D_{d}D_{s}}{D_{ds}}(\frac{1}{2}\theta^2_{E} -\theta_{E}\theta_{A} - \frac{1}{2}\theta^2_{E} + \theta_{E}\theta_{B})
 \end{equation}
  Therefore,
  
  \begin{equation}
  \Delta\text{t}(\beta)=\frac{(1+ \text{z}_{d})}{\text{c}}\frac{\text D_{d}D_{s}}{D_{ds}}\theta_{E}(\theta_{B}-\theta_{A})
  \end{equation}
  Thus, the time delay is proportional to the Einstein radius.\\
  
  
  
  \textbf{P.3.5: Minimum Dispersion Estimator}\\
  
  \noindent
  The minimum dispersion estimator is a simple, efficient and well tested method to estimate time delay from observed light curve\cite{manual}
  It helps to fine the difference between the curve at various delay times.
  Since it assume that light curves have the same shape but they are separated by time.\\
  
  
  \textbf{P3.6: The approximation of dispersion function near the minimum by a parabola.}\\
  
  The dispersion function near the minimum can be find by Taylor expanding of the functional function. i.e.
  
  \begin{equation}
  d^2(\lambda)=D^2(\lambda_{0}) +\frac{\text{d}D^2(\lambda_{0})}{\text{d}\lambda}(\lambda- \lambda_{0}) + \frac{\text{d}^2 D^2(\lambda_{0})}{\text{d}\lambda^2}(\lambda - \lambda_{0})^2 + 0
  \label{Equ:lambda}
  \end{equation}
  Where, $ \text{D}^2 $ is the dispersion function and $ \lambda $ is the time shift.
  Here high order terms are negligible compared to first and second order terms.\\
  
  The dispersion function is minimum at $ \lambda= \lambda_{0} $. Now the second term of equation \ref{Equ:lambda} goes to zero. i.e.
  
  \begin{equation}
  d^2(\lambda)=D^2(\lambda_{0}) + \frac{\text{d}^2 D^2(\lambda_{0})}{\text{d}\lambda^2}(\lambda - \lambda_{0})^2 
  \end{equation}
  it is parabolic in shape.