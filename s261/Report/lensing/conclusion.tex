\section{Conclusion}
In this chapter, we analyzed one pair of lens images, most importantly image separation and magnitude ratio are extracted. Combined with another independent observation, we determine the Hubble constant.

Just from the numerical value of the determined $H_0$, it is quite questionable already. Comparing to some other reputable measurements~\cite{Aghanim:2018eyx}\cite{Riess:2019cxk} for example, this value is roughly off by a factor of three. The origin of this discrepancy should be the lens model. SIS model is an axial symmetric model, where the sketch in~\cite{manual} objects are not properly aligned. The result we obtained is not realistic. 

In~\cite{Morgan:2003cf}, they have basically the same data: image separation $1''.2$, and $\sim 1:5$ flux ratio in $R$-band. However, because of the clear asymmetry, SIS model with a shear term is used. In the end, they are able to predict $\sim 33$ days time delay using the reasonable Hubble constant $H_0 = \SI{75}{\km\per\s\per\mega\parsec}$. So the inaccurate lens model should be blamed for this awful determination of $H_0$.
