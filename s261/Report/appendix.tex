\appendix
\chapter{Appendix}
\section{LIST file}\label{app:LIST}
\begin{tabular}{ccccccccccc}
   \toprule
COL & LINE & RMAG & FLUX & SKY & PIXELS & R & ELLIP & PA & PEAK & MFWHM\\
\midrule
\num{293.05} & \num{1859.53} & \num{19.30} & \num{191.3} & \num{0.15} & \num{80} & \num{3.51} & \num{0.019} & \num{82.5} & \num{3.98} & \num{7.96}\\
\num{435.95} & \num{1647.42} & \num{18.24} & \num{505.1} & \num{0.40} & \num{79} & \num{3.50} & \num{0.011} & \num{-63.1} & \num{10.49} & \num{8.12}\\
\num{974.61} & \num{1522.59} & \num{16.49} & \num{2531.7} & \num{2.10} & \num{79} & INDEF & \num{0.029} & \num{-38.3} & \num{52.22} & \num{8.16}\\
\num{652.00} & \num{1365.48} & \num{19.99} & \num{101.2} & \num{0.09} & \num{78} & \num{2.88} & \num{0.056} & \num{83.9} & \num{2.14} & \num{7.93}\\
\num{851.99} & \num{603.17} & \num{18.72} & \num{326.0} & \num{0.27} & \num{77} & \num{1.64} & \num{0.022} & \num{72.2} & \num{6.86} & \num{8.20}\\
\num{623.70} & \num{265.30} & \num{19.54} & \num{153.1} & \num{0.11} & \num{81} & \num{4.12} & \num{0.022} & \num{-13.9} & \num{3.18} & \num{7.88}\\
\num{279.06} & \num{1492.48} & \num{18.34} & \num{459.4} & \num{0.36} & \num{80} & \num{3.11} & \num{0.022} & \num{81.2} & \num{9.48} & \num{8.03}\\
\num{549.09} & \num{975.02} & \num{19.86} & \num{113.3} & \num{0.12} & \num{75} & INDEF & \num{0.015} & \num{68.6} & \num{2.32} & \num{8.71}\\
\bottomrule
\end{tabular}

\clearpage
\section{galfit.input}\label{app:galfit}
\lstinputlisting[language=Python]{../data&procedure/53/galfit.input}

\clearpage
\section{Output of galfit}\label{app:galfitOut}
First group of output is the fitting of two components (images), second group with only one component. Coordinates shown in curly brackets are the coordinates and their errors according to~\cite{galfitManual}. In our analysis, the variables/errors are interpreted as uncorrelated, so that a simple propagation of error formula can be used.
\lstinputlisting[language=Python]{../data&procedure/53/fit.log}

\clearpage
\section{Raw data of time delay estimate}\label{app:timdel}
\begin{table}%t1 
\caption{Photometry of \obj\ and of reference star~\#5, as in Fig.~\ref{lightcurve}. The Julian date corresponds to HJD-2~450~000~days. The five~points marked by an~asterisk are not used in the determination of the time delay.} 
\label{data} %
\centering \small 
\begin{tabular}{cccccccc} 
\hline\hline HJD & seeing [$''$] & mag A & $\sigma_A$ & mag B & $\sigma_B$ & mag star \#5 & $\sigma_{\rm star \#5}$ \\ \hline 3~133.412 & 1.5 & 0.273 & 0.006 & 2.244 & 0.016 & $-$0.175 & 0.013 \\ 3~134.362 & 1.1 & 0.291 & 0.002 & 2.241 & 0.011 & $-$0.166 & 0.004 \\ 3~135.385 & 0.9 & 0.282 & 0.004 & 2.268 & 0.012 & $-$0.177 & 0.008 \\ 3~142.386 & 1.3 & 0.287 & 0.002 & 2.273 & 0.007 & $-$0.167 & 0.003 \\ 3~146.415 & 1.2 & 0.276 & 0.001 & 2.264 & 0.006 & $-$0.178 & 0.002 \\ 3~150.406 & 0.9 & 0.264 & 0.001 & 2.300 & 0.006 & $-$0.176 & 0.001 \\ 3~152.367 & 1.0 & 0.279 & 0.004 & 2.231 & 0.011 & $-$0.172 & 0.008 \\ 3~154.407 & 0.9 & 0.267 & 0.002 & 2.252 & 0.007 & $-$0.175 & 0.003 \\ 3~159.364 & 1.1 & 0.253 & 0.003 & 2.254 & 0.018 & $-$0.174 & 0.005 \\ 3~163.315 & 1.7 & 0.249 & 0.002 & 2.301 & 0.010 & $-$0.173 & 0.003 \\ 3~166.425 & 1.0 & 0.248 & 0.002 & 2.244 & 0.008 & $-$0.177 & 0.003 \\ 3~169.393 & 1.1 & 0.222 & 0.003 & 2.317 & 0.011 & $-$0.189 & 0.006 \\ 3~171.430 & 1.5 & 0.234 & 0.002 & 2.284 & 0.008 & $-$0.172 & 0.004 \\ 3~173.298 & 1.1 & 0.233 & 0.001 & 2.275 & 0.006 & $-$0.168 & 0.002 \\ 3~175.300 & 1.1 & 0.230 & 0.001 & 2.279 & 0.006 & $-$0.178 & 0.001 \\ 3~189.265 & 1.1 & 0.208 & 0.007 & 2.236 & 0.023 & $-$0.172 & 0.012 \\ 3~191.360 & 1.1 & 0.215 & 0.003 & 2.255 & 0.018 & $-$0.167 & 0.004 \\ 3~193.304 & 1.3 & 0.194 & 0.006 & 2.246 & 0.018 & $-$0.166 & 0.012 \\ 3~195.252 & 1.0 & 0.217 & 0.002 & 2.239 & 0.007 & $-$0.170 & 0.004 \\ 3~203.314* & 1.4 & 0.186 & 0.001 & 2.345 & 0.007 & $-$0.173 & 0.002 \\ 3~209.256 & 1.0 & 0.189 & 0.003 & 2.198 & 0.010 & $-$0.167 & 0.007 \\ 3~213.256 & 1.0 & 0.166 & 0.003 & 2.190 & 0.010 & $-$0.183 & 0.007 \\ 3~216.310 & 1.7 & 0.149 & 0.006 & 2.189 & 0.018 & $-$0.163 & 0.011 \\ 3~219.281 & 1.6 & 0.151 & 0.004 & 2.198 & 0.018 & $-$0.155 & 0.008 \\ 3~221.234 & 1.0 & 0.178 & 0.002 & 2.241 & 0.009 & $-$0.179 & 0.003 \\ 3~224.209 & 1.2 & 0.140 & 0.005 & 2.228 & 0.015 & $-$0.166 & 0.010 \\ 3~232.187 & 1.1 & 0.215 & 0.003 & 2.198 & 0.009 & $-$0.168 & 0.005 \\ 3~246.163 & 1.3 & 0.133 & 0.013 & 2.148 & 0.035 & $-$0.142 & 0.026 \\ 3~248.138* & 1.5 & 0.172 & 0.004 & 2.054 & 0.013 & $-$0.114 & 0.007 \\ 3~254.125 & 0.9 & 0.156 & 0.004 & 2.139 & 0.013 & $-$0.160 & 0.007 \\ 3~256.135 & 1.0 & 0.188 & 0.001 & 2.181 & 0.005 & $-$0.173 & 0.003 \\ 3~261.115 & 1.5 & 0.203 & 0.011 & 2.153 & 0.029 & $-$0.178 & 0.017 \\ 3~271.109 & 1.3 & 0.191 & 0.005 & 2.167 & 0.014 & $-$0.183 & 0.009 \\ 3~273.103 & 1.2 & 0.163 & 0.007 & 2.117 & 0.020 & $-$0.176 & 0.013 \\ 3~277.104 & 1.0 & 0.190 & 0.004 & 2.179 & 0.014 & $-$0.191 & 0.007 \\ 3~281.106 & 1.4 & 0.146 & 0.006 & 2.151 & 0.016 & $-$0.167 & 0.011 \\ 3~283.100 & 1.2 & 0.146 & 0.004 & 2.129 & 0.012 & $-$0.164 & 0.008 \\ 3~430.540 & 0.9 & 0.167 & 0.002 & 2.078 & 0.009 & $-$0.135 & 0.003 \\ 3~434.546 & 0.8 & 0.175 & 0.007 & 2.088 & 0.019 & $-$0.143 & 0.009 \\ 3~461.499 & 0.9 & 0.166 & 0.003 & 2.138 & 0.009 & $-$0.150 & 0.005 \\ 3~475.491 & 0.9 & 0.175 & 0.004 & 2.120 & 0.012 & $-$0.179 & 0.008 \\ 3~482.443 & 1.1 & 0.177 & 0.002 & 2.124 & 0.011 & $-$0.149 & 0.004 \\ 3~500.415 & 1.3 & 0.206 & 0.004 & 2.147 & 0.011 & $-$0.168 & 0.008 \\ 3~507.348 & 1.0 & 0.177 & 0.005 & 2.106 & 0.012 & $-$0.153 & 0.009 \\ 3~508.403* & 0.9 & 0.199 & 0.003 & 2.222 & 0.009 & $-$0.171 & 0.006 \\ 3~511.303 & 1.0 & 0.172 & 0.007 & 2.153 & 0.018 & $-$0.151 & 0.012 \\ 3~517.383 & 0.8 & 0.163 & 0.004 & 2.146 & 0.012 & $-$0.168 & 0.007 \\ 3~524.391 & 0.9 & 0.194 & 0.002 & 2.118 & 0.006 & $-$0.183 & 0.003 \\ 3~533.412* & 1.6 & 0.193 & 0.002 & 2.002 & 0.007 & $-$0.171 & 0.004 \\ 3~540.345 & 1.0 & 0.194 & 0.003 & 2.124 & 0.011 & $-$0.168 & 0.006 \\ 3~542.323 & 1.1 & 0.172 & 0.007 & 2.132 & 0.020 & $-$0.194 & 0.014 \\ 3~552.291 & 0.9 & 0.201 & 0.001 & 2.149 & 0.005 & $-$0.179 & 0.001 \\ 3~556.295 & 0.9 & 0.204 & 0.003 & 2.185 & 0.008 & $-$0.176 & 0.005 \\ 3~559.280 & 1.0 & 0.181 & 0.002 & 2.134 & 0.007 & $-$0.160 & 0.004 \\ 3~564.295* & 1.3 & 0.228 & 0.001 & 2.300 & 0.007 & $-$0.167 & 0.003 \\ 3~570.247 & 1.0 & 0.192 & 0.005 & 2.094 & 0.015 & $-$0.165 & 0.010 \\  \hline
\end{tabular} 
\end{table} 

\begin{table}%t1 
\begin{tabular}{cccccccc} 
\hline\hline HJD & seeing [$''$] & mag A & $\sigma_A$ & mag B & $\sigma_B$ & mag star \#5 & $\sigma_{\rm star \#5}$ \\ \hline
3~575.323 & 1.3 & 0.198 & 0.010 & 2.084 & 0.026 & $-$0.172 & 0.019 \\ 3~576.264 & 1.0 & 0.211 & 0.001 & 2.195 & 0.008 & $-$0.181 & 0.002\\
3~578.275 & 1.2 & 0.189 & 0.002 & 2.133 & 0.007 & $-$0.177 & 0.003 \\ 3~581.284 & 1.2 & 0.190 & 0.005 & 2.176 & 0.012 & $-$0.188 & 0.009 \\ 3~611.225 & 1.2 & 0.188 & 0.003 & 2.161 & 0.008 & $-$0.165 & 0.005 \\ 3~613.201 & 1.5 & 0.175 & 0.007 & 2.156 & 0.018 & $-$0.180 & 0.013 \\ \hline 
\end{tabular} 
\end{table} 


\clearpage
\section{Input parameters of program tdel}
\begin{table}[ht]
	\centering
	\begin{tabular}{cc}
		\toprule
		parameter & value \\	
		\midrule
		\verb|Delay_guess| & 30 \\
		\verb|Delay_Min| & 10 \\
		\verb|Delay_Max| & 50 \\
		\verb|Delay_nbin| & 10 \\
		\verb|mmag_Max| & -2.5 \\
		\verb|mmag_Min| & 2.5 \\
		\verb|mmag_nbin| & 50 \\
		\verb|UseMC| &	0 \\
		\verb|NMC| & 0\\
		\verb|MinGapLength| & 100\\
		\bottomrule
	\end{tabular} 
	\caption{Initial content of tdel.param. Some other inputs are not shown here, since they remain unchanged from default values. Later Monte Carlo is turned on with UseMC $1$ and NMC $500$.}
	\label{tab:tdelInParam}
\end{table}

\clearpage
\section{Raw data of Monte Carlo}\label{app:MCout}
\begin{table}[htpb]
	\centering
\begin{tabular}{cc}
\toprule
time delay[days] & probability \\
\midrule
\num{30.5613} & \num{0.0547047}\\
\num{31.3402} & \num{0.0707943}\\
\num{32.119} & \num{0.119063}\\
\num{32.8979} & \num{0.189858}\\
\num{33.6767} & \num{0.205947}\\
\num{34.4556} & \num{0.18664}\\
\num{35.2344} & \num{0.160896}\\
\num{36.0133} & \num{0.112627}\\
\num{36.7921} & \num{0.0740123}\\
\num{37.5709} & \num{0.0225255}\\
\num{38.3498} & \num{0.0547047}\\
\num{39.1286} & \num{0.0160896}\\
\num{39.9075} & \num{0.00643585}\\
\num{40.6863} & \num{0.00321792}\\
\num{41.4652} & \num{0}\\
\num{42.244} & \num{0.00321792}\\
\num{43.0229} & \num{0}\\
\num{43.8017} & \num{0}\\
\num{44.5806} & \num{0.00321792}\\
\num{45.3594} & \num{0}\\
\bottomrule
\end{tabular}	
	\caption{Output of Monte Carlo method}
\end{table}



