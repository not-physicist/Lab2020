\begin{abstract}
   In this experiment, we investigate nuclear properties of cobalt nuclei via angular correlation. With scintillation detectors, we measure number of coincidence in dependence on angle between two photons. There is strong anisotropy in coincidence rate present. The measured data are fitted with the theoretical prediction function and calculated the angular correlation coefficients: $ A_{22}= \num{0.0849 +- 0.0057} $ and $A_{44}=\num{-0.0002 +- 0.0063}  $. Closer investigation reveals that measured values differ from prediction by $4\sigma$. Other types of cascade are excluded almost $100\%$.
\end{abstract}

\section{Introduction}
In a cascade gamma decay of nuclei a non-isotropic angular distribution can be measured due to non-equilibrium spin state; however, in thermal equilibrium net orientation of all spin states is zero due to gamma ray distribution of nuclei is isotropic. Thus, the angular correlation between two gamma rays emitted during the cascaded gamma decay of the nucleus is due to the unequal spin states distribution in the intermediate states ~\cite{descr}.
The main motive of the experiment is set up the experiment and investigate nuclear properties of $ ^{60}Co $ via angular correlation of $ \gamma-\gamma $ cascades ~\cite{descr}.
