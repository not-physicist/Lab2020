\begin{abstract}
   In this experiment, we investigate nuclear properties of cobalt nuclei via angular correlation. we have set up gain of main amplifier ($ A_{1} $ and $ A_{2} $ ) and Constant Fraction Discriminator(CFD) threshold has to be adjusted to filtered events. Afterward, we inert a fixed delay into one fast branch and a variable delay in to another one.Then, We picked resolving time( ns) for the first measurement. After setting up all the apparatus, we measured the coincidence with the help two detectors with varying angle between them. Then, we fitted the measured data with the theoretical prediction function and calculated the angular correlation coefficients $ A_{22}= 0.125\pm 0.031 $ and $A_{44}=-0.001\pm 0.029  $. The non-zero values of these two angular correlation coefficients, importantly, provide strong evidence for an angular correlation between two gamma rays emitted in cascade of $ ^{60}Co $. These coefficients almost agree with the theoretical prediction within the error limit; however, there is little bit deviation from theoretical value, maybe,  due to the errors in our measurements.
\end{abstract}

\section{Introduction}
In a cascade gamma decay of nuclei a non-isotropic angular distribution can be measured due to non-equilibrium spin state; however, in thermal equilibrium net orientation of all spin states is zero due to gamma ray distribution of nuclei is isotropic. Thus, the angular correlation between two gamma rays emitted during the cascaded gamma decay of
the nucleus is due to the unequal spin states distribution in the intermediate states \cite{descr}.
The main motive of the experiment is set up the experiment and investigate nuclear properties of $ ^60Co $ via angular correlation of $ \gamma-\gamma $ cascades\cite{descr}.