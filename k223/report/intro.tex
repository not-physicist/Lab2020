\begin{abstract}
   In this experiment, we investigate nuclear properties of cobalt nuclei via angular correlation. Number of coincidence is measured in dependence on angle between two photons. There is strong anisotropy present. The measured data are fitted with the theoretical prediction function of $420$ cascade and calculated the angular correlation coefficients: $ A_{22}= \num{0.0849 +- 0.0057} $ and $A_{44}=\num{-0.0002 +- 0.0063}  $. Closer investigation reveals that measured values differ from theoretical prediction by $4\sigma$. Despite this, other types of cascade are excluded almost $100\%$.
\end{abstract}

\section{Introduction}
Angular correlation from cascades can appear because of the unequal spin states distribution in the intermediate states. By choosing one photon from cascades one puts constraints in angular distribution of second photon~\cite{descr}. For detection of gamma rays, scintillator and photomultiplier are used. In order to extract both energy and timing accurately, so-called "fast-slow principle" is used. The main parts of the experiment are setting up the experiment and investigating nuclear properties of $ ^{60}Co $ via angular correlation of $ \gamma$-$\gamma $ cascades~\cite{descr}.

Section~\ref{sec:theory} briefly talks about theories underlying angular correlation of cascades. Section~\ref{sec:setup} explains functionality of electronics used in setup. Tasks for preparation are in section~\ref{sec:task}. In section~\ref{sec:procedure}, one can find calibration and preparation processes. Data analysis is presented in section~\ref{sec:analysis}. Conclusion is in section~\ref{sec:conclusion}
