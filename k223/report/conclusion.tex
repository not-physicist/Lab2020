\section{Discussion And Conclusion}
In our experiment, we investigate nuclear properties of cobalt nuclei via angular correlation for that, firstly, we have set up gain of main amplifier ($ A_{1} $ and $ A_{2} $ ) and upper limit for the gain used in the set up. Then Constant Fraction Discriminator(CFD) threshold has to be adjusted to filtered events. Afterward,we inert a fixed delay into one fast branch and a variable delay in to another one.Then,We picked resolving time (25 ns) for the first measurement. Furthermore, to adjust the timing of the fast coincidence (FC) unit connect a gate and delay generator to the output of FC.
After setting up all the apparatus,we measured the coincidence using two detectors--one of them is fixed and another one is movable-- with varying angle between them. Then, we fitted the measured data with the theoretical prediction function and calculated the angular correlation coefficients $ A_{22}= 0.125\pm 0.031 $ and $A_{44}=-0.001\pm 0.029  $. The non zero values of these two angular correlation coefficients, importantly, provide strong evidence for an angular correlation between two gamma rays emitted in cascade of $ ^{60}Co $. These coefficients almost agree with the theoretical prediction within the error limit; however, there is little bit deviation from theoretical value, maybe,  due to the errors in our measurements. 