\section{Conclusion}\label{sec:conclusion}
In this experiment, we measured angular correlation between two photons emitted from ${}^{60}\text{Co}$. Then, the data gets fitted with the theoretical prediction function and  the angular correlation coefficients are calculated:  $ A_{22}= \num{0.0849 +- 0.0057} $ and $A_{44}=\num{-0.0002 +- 0.0063} $. The non-zero values of these two angular correlation coefficients provide strong evidence for an angular correlation between two gamma rays emitted in 420 cascade of $ ^{60}\text{Co} $. Via curve fitting, we can conclude that other types of cascade (020,121, 220, and 320) are basically $100\%$ excluded.

Further investigation reveals that measured values differ from prediction by $4\sigma$ due to unknown systematic errors in the setup. One of these systematic errors could very well be in electronics. Geometry of the setup could be measured and investigated more. As mentioned before, the resolving time can be lowered a bit. But it should not have huge impact on determination of angular correlation coefficients. 

\section{Acknowledgement}
We would heartly acknowledge the Bonn-Cologne Graduate School of Physics and Astronomy (BCGS) for the opportunity to perform the experiment. We would like to express our gratitude and appreciation for Dr. Christian Honisch for the tutoring, encouragement, and guidance throughout the experiment.
