\section{Discussion And Conclusion}
In our experiment, we investigate nuclear properties of cobalt nuclei via angular correlation. After setting up all the apparatus, we measured the coincidence using two detectors one of them is fixed and another one is movable with varying angle between them. Then, we fitted the measured data with the theoretical prediction function and calculated the angular correlation coefficients:  $ A_{22}= \num{0.0849 +- 0.0057} $ and $A_{44}=\num{-0.0002 +- 0.0063} $. The non zero values of these two angular correlation coefficients, importantly, provide strong evidence for an angular correlation between two gamma rays emitted in 010 cascade of $ ^{60}Co $. Further investigation reveals that measured values differ from prediction by $4\sigma$ due to systematic error in the set up. On the basis of found values of B and C, we can concluded that, Other types of cascade (020,121, 220, and 320) are excluded almost $100\%$. 
